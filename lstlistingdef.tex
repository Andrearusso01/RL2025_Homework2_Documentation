\usepackage{listings}
\usepackage{xcolor}

\definecolor{codegreen}{rgb}{0,0.6,0}
\definecolor{codegray}{rgb}{0.5,0.5,0.5}
\definecolor{codepurple}{rgb}{0.58,0,0.82}
\definecolor{backcolour}{rgb}{0.95,0.95,0.92}
\definecolor{mygreen}{rgb}{0,0.6,0}
\definecolor{mygray}{rgb}{0.5,0.5,0.5}
\definecolor{mymauve}{rgb}{0.58,0,0.82}
\definecolor{editorLightGray}{cmyk}{0.05, 0.05, 0.05, 0.1}
\definecolor{editorGray}{cmyk}{0.6, 0.55, 0.55, 0.2}
\definecolor{editorPurple}{cmyk}{0.5, 1, 0, 0}
\definecolor{editorWhite}{cmyk}{0, 0, 0, 0}
\definecolor{editorBlack}{cmyk}{1, 1, 1, 1}
\definecolor{editorOrange}{cmyk}{0, 0.8, 1, 0}
\definecolor{editorBlue}{cmyk}{1, 0.6, 0, 0}
\definecolor{editorPink}{cmyk}{0, 1, 0, 0}

% \lstset{
% 	language=html,
% 	columns=flexible,
% 	tagstyle=\color{editorBlue},
% 	basicstyle={\small\ttfamily},
% 	identifierstyle=\color{editorOrange},
% 	keywordstyle=\color{editorPink},
% 	commentstyle=\color{editorGray},
% 	stringstyle=\color{editorPurple}
% }

\lstdefinestyle{bash}{
    language=bash,
	backgroundcolor=\color{white},   % choose the background color; you must add \usepackage{color} or \usepackage{xcolor}; should come as last argument
	basicstyle=\ttfamily,        % the size of the fonts that are used for the code
	columns=flexible,
	breakatwhitespace=false,         % sets if automatic breaks should only happen at whitespace
	breaklines=true,                 % sets automatic line breaking
	captionpos=b,                    % sets the caption-position to bottom
	%commentstyle=\color{mygreen},    % comment style
	deletekeywords={...},            % if you want to delete keywords from the given language
	escapeinside={\%*}{*)},          % if you want to add LaTeX within your code
	extendedchars=true,              % lets you use non-ASCII characters; for 8-bits encodings only, does not work with UTF-8
	firstnumber=1000,                % start line enumeration with line 1000
	frame=none,	                 % adds a frame around the code
	identifierstyle=\color{black},
	keepspaces=true,                 % keeps spaces in text, useful for keeping indentation of code (possibly needs columns=flexible)
	keywordstyle=\color{black},       % keyword style
	%language=Octave,                 % the language of the code
	%morekeywords={*,...},            % if you want to add more keywords to the set
	%numbers=left,                    % where to put the line-numbers; possible values are (none, left, right)
	%numbersep=5pt,                   % how far the line-numbers are from the code
	%numberstyle=\tiny\color{mygray}, % the style that is used for the line-numbers
	rulecolor=\color{black},         % if not set, the frame-color may be changed on line-breaks within not-black text (e.g. comments (green here))
	showspaces=false,                % show spaces everywhere adding particular underscores; it overrides 'showstringspaces'
	showstringspaces=false,          % underline spaces within strings only
	showtabs=false,                  % show tabs within strings adding particular underscores
	stepnumber=2,                    % the step between two line-numbers. If it's 1, each line will be numbered
	stringstyle=\color{mymauve},     % string literal style
	tabsize=2,	                     % sets default tabsize to 2 spaces
	%title=\lstname                   % show the filename of files included with \lstinputlisting; also try caption instead of title
	literate={-}{{-}}1 
}

 \lstdefinestyle{html}{
    language=html,
	backgroundcolor=\color{white},   % choose the background color; you must add \usepackage{color} or \usepackage{xcolor}; should come as last argument
    alsoletter={<>=-},
    otherkeywords={
            % HTML tags
            />,
            <launch>, </launch>,
            <node>, <node, </node>,
            <arg, <param, <remap, <include
        },
	basicstyle=\small\ttfamily,        % the size of the fonts that are used for the code
	columns=flexible,
	breakatwhitespace=false,         % sets if automatic breaks should only happen at whitespace
	breaklines=true,                 % sets automatic line breaking
	captionpos=b,                    % sets the caption-position to bottom
	commentstyle=\color{mygreen},    % comment style
	deletekeywords={...},            % if you want to delete keywords from the given language
	escapeinside={\%*}{*)},          % if you want to add LaTeX within your code
	extendedchars=true,              % lets you use non-ASCII characters; for 8-bits encodings only, does not work with UTF-8
	firstnumber=1000,                % start line enumeration with line 1000
	frame=none,	                 % adds a frame around the code
	identifierstyle=\color{black},
	keepspaces=true,                 % keeps spaces in text, useful for keeping indentation of code (possibly needs columns=flexible)
	keywordstyle=\color{blue},       % keyword style
	%language=Octave,                 % the language of the code
	%morekeywords={*,...},            % if you want to add more keywords to the set
	%numbers=left,                    % where to put the line-numbers; possible values are (none, left, right)
	%numbersep=5pt,                   % how far the line-numbers are from the code
	%numberstyle=\tiny\color{mygray}, % the style that is used for the line-numbers
	rulecolor=\color{black},         % if not set, the frame-color may be changed on line-breaks within not-black text (e.g. comments (green here))
	showspaces=false,                % show spaces everywhere adding particular underscores; it overrides 'showstringspaces'
	showstringspaces=false,          % underline spaces within strings only
	showtabs=false,                  % show tabs within strings adding particular underscores
	stepnumber=2,                    % the step between two line-numbers. If it's 1, each line will be numbered
	stringstyle=\color{mymauve},     % string literal style
	tabsize=2,	                     % sets default tabsize to 2 spaces
	%title=\lstname                   % show the filename of files included with \lstinputlisting; also try caption instead of title
	%literate={~}{$\sim$}{1},
	literate={-}{-}1 
}

\lstdefinestyle{cpp}{
        language=C++,
        basicstyle=\ttfamily,
        keywordstyle=\color{blue}\ttfamily,
        stringstyle=\color{red}\ttfamily,
        commentstyle=\color{green}\ttfamily
        %escapeinside={\%*}{*)},
        otherkeywords={include}
        %morecomment=[l][\color{magenta}]{\#},
        %literate={#}{\#}1
}

\lstdefinestyle{makefile}{
    basicstyle=\ttfamily,%
    stringstyle=\itshape\color{magenta},%
    showstringspaces=false,%
    keywordstyle=\bfseries\color{keycolor},%
    commentstyle=\color{blue}\slshape,%
    framexleftmargin=1mm,%
    otherkeywords={.SUFFIXES},
    morekeywords={SUFFIX, CPP_,},
    moredelim=[is][\color{bleu}]{/*}{*/},
    morecomment=[l][commentstyle]{\#},%
    emphstyle={\color{vimvert}},%
    moredelim=[s][\color{vimvert}]{\$(}{)}%
}


\lstset{
    style=bash
	}	